% Options for packages loaded elsewhere
\PassOptionsToPackage{unicode}{hyperref}
\PassOptionsToPackage{hyphens}{url}
%
\documentclass[
]{article}
\usepackage{amsmath,amssymb}
\usepackage{lmodern}
\usepackage{ifxetex,ifluatex}
\ifnum 0\ifxetex 1\fi\ifluatex 1\fi=0 % if pdftex
  \usepackage[T1]{fontenc}
  \usepackage[utf8]{inputenc}
  \usepackage{textcomp} % provide euro and other symbols
\else % if luatex or xetex
  \usepackage{unicode-math}
  \defaultfontfeatures{Scale=MatchLowercase}
  \defaultfontfeatures[\rmfamily]{Ligatures=TeX,Scale=1}
\fi
% Use upquote if available, for straight quotes in verbatim environments
\IfFileExists{upquote.sty}{\usepackage{upquote}}{}
\IfFileExists{microtype.sty}{% use microtype if available
  \usepackage[]{microtype}
  \UseMicrotypeSet[protrusion]{basicmath} % disable protrusion for tt fonts
}{}
\makeatletter
\@ifundefined{KOMAClassName}{% if non-KOMA class
  \IfFileExists{parskip.sty}{%
    \usepackage{parskip}
  }{% else
    \setlength{\parindent}{0pt}
    \setlength{\parskip}{6pt plus 2pt minus 1pt}}
}{% if KOMA class
  \KOMAoptions{parskip=half}}
\makeatother
\usepackage{xcolor}
\IfFileExists{xurl.sty}{\usepackage{xurl}}{} % add URL line breaks if available
\IfFileExists{bookmark.sty}{\usepackage{bookmark}}{\usepackage{hyperref}}
\hypersetup{
  pdftitle={Hypothesis Testing},
  pdfauthor={Edwin Lo, GWU Intro to Data Science DATS 6101},
  hidelinks,
  pdfcreator={LaTeX via pandoc}}
\urlstyle{same} % disable monospaced font for URLs
\usepackage[margin=1in]{geometry}
\usepackage{color}
\usepackage{fancyvrb}
\newcommand{\VerbBar}{|}
\newcommand{\VERB}{\Verb[commandchars=\\\{\}]}
\DefineVerbatimEnvironment{Highlighting}{Verbatim}{commandchars=\\\{\}}
% Add ',fontsize=\small' for more characters per line
\usepackage{framed}
\definecolor{shadecolor}{RGB}{248,248,248}
\newenvironment{Shaded}{\begin{snugshade}}{\end{snugshade}}
\newcommand{\AlertTok}[1]{\textcolor[rgb]{0.94,0.16,0.16}{#1}}
\newcommand{\AnnotationTok}[1]{\textcolor[rgb]{0.56,0.35,0.01}{\textbf{\textit{#1}}}}
\newcommand{\AttributeTok}[1]{\textcolor[rgb]{0.77,0.63,0.00}{#1}}
\newcommand{\BaseNTok}[1]{\textcolor[rgb]{0.00,0.00,0.81}{#1}}
\newcommand{\BuiltInTok}[1]{#1}
\newcommand{\CharTok}[1]{\textcolor[rgb]{0.31,0.60,0.02}{#1}}
\newcommand{\CommentTok}[1]{\textcolor[rgb]{0.56,0.35,0.01}{\textit{#1}}}
\newcommand{\CommentVarTok}[1]{\textcolor[rgb]{0.56,0.35,0.01}{\textbf{\textit{#1}}}}
\newcommand{\ConstantTok}[1]{\textcolor[rgb]{0.00,0.00,0.00}{#1}}
\newcommand{\ControlFlowTok}[1]{\textcolor[rgb]{0.13,0.29,0.53}{\textbf{#1}}}
\newcommand{\DataTypeTok}[1]{\textcolor[rgb]{0.13,0.29,0.53}{#1}}
\newcommand{\DecValTok}[1]{\textcolor[rgb]{0.00,0.00,0.81}{#1}}
\newcommand{\DocumentationTok}[1]{\textcolor[rgb]{0.56,0.35,0.01}{\textbf{\textit{#1}}}}
\newcommand{\ErrorTok}[1]{\textcolor[rgb]{0.64,0.00,0.00}{\textbf{#1}}}
\newcommand{\ExtensionTok}[1]{#1}
\newcommand{\FloatTok}[1]{\textcolor[rgb]{0.00,0.00,0.81}{#1}}
\newcommand{\FunctionTok}[1]{\textcolor[rgb]{0.00,0.00,0.00}{#1}}
\newcommand{\ImportTok}[1]{#1}
\newcommand{\InformationTok}[1]{\textcolor[rgb]{0.56,0.35,0.01}{\textbf{\textit{#1}}}}
\newcommand{\KeywordTok}[1]{\textcolor[rgb]{0.13,0.29,0.53}{\textbf{#1}}}
\newcommand{\NormalTok}[1]{#1}
\newcommand{\OperatorTok}[1]{\textcolor[rgb]{0.81,0.36,0.00}{\textbf{#1}}}
\newcommand{\OtherTok}[1]{\textcolor[rgb]{0.56,0.35,0.01}{#1}}
\newcommand{\PreprocessorTok}[1]{\textcolor[rgb]{0.56,0.35,0.01}{\textit{#1}}}
\newcommand{\RegionMarkerTok}[1]{#1}
\newcommand{\SpecialCharTok}[1]{\textcolor[rgb]{0.00,0.00,0.00}{#1}}
\newcommand{\SpecialStringTok}[1]{\textcolor[rgb]{0.31,0.60,0.02}{#1}}
\newcommand{\StringTok}[1]{\textcolor[rgb]{0.31,0.60,0.02}{#1}}
\newcommand{\VariableTok}[1]{\textcolor[rgb]{0.00,0.00,0.00}{#1}}
\newcommand{\VerbatimStringTok}[1]{\textcolor[rgb]{0.31,0.60,0.02}{#1}}
\newcommand{\WarningTok}[1]{\textcolor[rgb]{0.56,0.35,0.01}{\textbf{\textit{#1}}}}
\usepackage{graphicx}
\makeatletter
\def\maxwidth{\ifdim\Gin@nat@width>\linewidth\linewidth\else\Gin@nat@width\fi}
\def\maxheight{\ifdim\Gin@nat@height>\textheight\textheight\else\Gin@nat@height\fi}
\makeatother
% Scale images if necessary, so that they will not overflow the page
% margins by default, and it is still possible to overwrite the defaults
% using explicit options in \includegraphics[width, height, ...]{}
\setkeys{Gin}{width=\maxwidth,height=\maxheight,keepaspectratio}
% Set default figure placement to htbp
\makeatletter
\def\fps@figure{htbp}
\makeatother
\setlength{\emergencystretch}{3em} % prevent overfull lines
\providecommand{\tightlist}{%
  \setlength{\itemsep}{0pt}\setlength{\parskip}{0pt}}
\setcounter{secnumdepth}{-\maxdimen} % remove section numbering
\usepackage{booktabs}
\usepackage{longtable}
\usepackage{array}
\usepackage{multirow}
\usepackage{wrapfig}
\usepackage{float}
\usepackage{colortbl}
\usepackage{pdflscape}
\usepackage{tabu}
\usepackage{threeparttable}
\usepackage{threeparttablex}
\usepackage[normalem]{ulem}
\usepackage{makecell}
\usepackage{xcolor}
\ifluatex
  \usepackage{selnolig}  % disable illegal ligatures
\fi

\title{Hypothesis Testing}
\author{Edwin Lo, GWU Intro to Data Science DATS 6101}
\date{2022-02-14}

\begin{document}
\maketitle

{
\setcounter{tocdepth}{3}
\tableofcontents
}
\hypertarget{hypothesis-testing}{%
\section{Hypothesis Testing}\label{hypothesis-testing}}

We performed estimations (Z-intervals and T-intervals) in our last
exercise. This time, we are performing hypothesis testings.

\hypertarget{preparation}{%
\subsection{Preparation}\label{preparation}}

Continuing from last time where we found the Z- and T- intervals\ldots{}

\begin{Shaded}
\begin{Highlighting}[]
\FunctionTok{getwd}\NormalTok{()}
\CommentTok{\# mlb \textless{}{-} data.frame(read.csv("BaseballHeightWeight.csv", header = TRUE))}
\NormalTok{mlb }\OtherTok{\textless{}{-}} \FunctionTok{read.csv}\NormalTok{(}\StringTok{"BaseballHeightWeight.csv"}\NormalTok{, }\AttributeTok{header =} \ConstantTok{TRUE}\NormalTok{)}
\FunctionTok{str}\NormalTok{(mlb)}
\FunctionTok{head}\NormalTok{(mlb)}
\FunctionTok{colnames}\NormalTok{(mlb)[}\DecValTok{4}\SpecialCharTok{:}\DecValTok{6}\NormalTok{]}\OtherTok{=}\FunctionTok{c}\NormalTok{(}\StringTok{"height"}\NormalTok{,}\StringTok{"weight"}\NormalTok{,}\StringTok{"age"}\NormalTok{)}
\CommentTok{\# The line above produces no output. To see the changes, use str() or head()}
\end{Highlighting}
\end{Shaded}

\begin{Shaded}
\begin{Highlighting}[]
\FunctionTok{xkabledplyhead}\NormalTok{(mlb)}
\end{Highlighting}
\end{Shaded}

\begin{table}

\caption{\label{tab:unnamed-chunk-2}Head}
\begin{tabular}[t]{l|l|l|r|r|r}
\hline
Name & Team & Position & height & weight & age\\
\hline
Adam\_Donachie & BAL & Catcher & 74 & 180 & 23.0\\
\hline
Paul\_Bako & BAL & Catcher & 74 & 215 & 34.7\\
\hline
Ramon\_Hernandez & BAL & Catcher & 72 & 210 & 30.8\\
\hline
Kevin\_Millar & BAL & First\_Baseman & 72 & 210 & 35.4\\
\hline
Chris\_Gomez & BAL & First\_Baseman & 73 & 188 & 35.7\\
\hline
\end{tabular}
\end{table}

\begin{Shaded}
\begin{Highlighting}[]
\CommentTok{\# xkabledplytail(mlb, 3)}
\end{Highlighting}
\end{Shaded}

Again, first take note of the mean height and weight for the entire mlb
population is height: 73.697 inches, weight: 201.69 lbs. Then redo what
we had last time with Z- and T-intervals, but put in the appropriate
null hypothesis for the Z- and T-tests.

\begin{Shaded}
\begin{Highlighting}[]
\FunctionTok{set.seed}\NormalTok{(}\DecValTok{123}\NormalTok{) }\CommentTok{\# just so that everyone have the same sample for comparison}
\NormalTok{mlbsample }\OtherTok{=}\NormalTok{ mlb[ }\FunctionTok{sample}\NormalTok{(}\FunctionTok{nrow}\NormalTok{(mlb),}\DecValTok{30}\NormalTok{), ]}
\FunctionTok{str}\NormalTok{(mlbsample)}
\FunctionTok{head}\NormalTok{(mlbsample)}
\FunctionTok{format}\NormalTok{(}\FunctionTok{mean}\NormalTok{(mlbsample}\SpecialCharTok{$}\NormalTok{height), }\AttributeTok{digits=}\DecValTok{4}\NormalTok{)}
\FunctionTok{format}\NormalTok{(}\FunctionTok{mean}\NormalTok{(mlbsample}\SpecialCharTok{$}\NormalTok{weight, }\AttributeTok{na.rm =} \ConstantTok{TRUE}\NormalTok{), }\AttributeTok{digits=}\DecValTok{5}\NormalTok{)}
\end{Highlighting}
\end{Shaded}

\hypertarget{z-test}{%
\subsection{Z-Test}\label{z-test}}

So let's say we want to test is the mean height is 72.1 inches like last
year as our null hypothesis \(H_0: \mu = 72.1\) inches, which was the
previous year's average. (Notice the use of latex inline formatting of
mathematical typesetting, enclosed with a pair of dollar sign \$ \$ for
the inline R-code.)

\begin{Shaded}
\begin{Highlighting}[]
\FunctionTok{loadPkg}\NormalTok{(}\StringTok{"BSDA"}\NormalTok{) }\CommentTok{\# for z.test}
\NormalTok{ztest95\_2tail }\OtherTok{=} \FunctionTok{z.test}\NormalTok{(}\AttributeTok{x=}\NormalTok{mlbsample}\SpecialCharTok{$}\NormalTok{height, }\AttributeTok{mu=}\FloatTok{72.1}\NormalTok{, }\AttributeTok{sigma.x =} \FloatTok{2.31}\NormalTok{) }\CommentTok{\# }
\NormalTok{ztest95\_2tail}
\NormalTok{ztest95\_right }\OtherTok{=} \FunctionTok{z.test}\NormalTok{(}\AttributeTok{x=}\NormalTok{mlbsample}\SpecialCharTok{$}\NormalTok{height, }\AttributeTok{mu=}\FloatTok{72.1}\NormalTok{, }\AttributeTok{sigma.x =} \FloatTok{2.31}\NormalTok{, }\AttributeTok{alternative =} \StringTok{"greater"}\NormalTok{) }\CommentTok{\# }
\NormalTok{ztest95\_right}
\CommentTok{\# ztest99 = z.test(x=mlbsample$height, mu=72.1, sigma.x = 2.31, conf.level=0.99 )}
\CommentTok{\# ztest99}
\CommentTok{\# ztest50 = z.test(x=mlbsample$height, mu=72.1, sigma.x = 2.31, conf.level=0.50 )}
\CommentTok{\# ztest50}
\end{Highlighting}
\end{Shaded}

\begin{Shaded}
\begin{Highlighting}[]
\CommentTok{\# force this one result to always show, regardless of global settings}
\NormalTok{ztest95\_right}
\end{Highlighting}
\end{Shaded}

\begin{verbatim}
## 
##  One-sample z-Test
## 
## data:  mlbsample$height
## z = 4, p-value = 7e-05
## alternative hypothesis: true mean is greater than 72.1
## 95 percent confidence interval:
##  73 NA
## sample estimates:
## mean of x 
##      73.7
\end{verbatim}

You can also try and see it for yourself, the optional argument for
\texttt{conf.level=0.50} for example does not change the outcome for the
p-value or the conclusion of the test. It only changes the confidence
interval of the outcome.

In cases that you want to determine the confidence Z-interval
(estimation) and Z-test as well, you can do so by supplying the
\texttt{conf.level} value in the function, and get both results in one
step. Typically, you would want to set the conf.level for the interval
to be 1 minus the \(\alpha\) level for the test to obtain the
appropriate corresponding results.

Now let us try the left- and right-tailed tests

\begin{Shaded}
\begin{Highlighting}[]
\NormalTok{ztestrighttail }\OtherTok{=} \FunctionTok{z.test}\NormalTok{(}\AttributeTok{x=}\NormalTok{mlbsample}\SpecialCharTok{$}\NormalTok{height, }\AttributeTok{mu=}\FloatTok{72.1}\NormalTok{, }\AttributeTok{sigma.x =} \FloatTok{2.31}\NormalTok{, }\AttributeTok{alternative =} \StringTok{\textquotesingle{}greater\textquotesingle{}}\NormalTok{ )}
\NormalTok{ztestrighttail  }\CommentTok{\# p{-}value is small, reject null, adopt alternative mu is greater than 73.2}
\NormalTok{ztest99lefttail }\OtherTok{=} \FunctionTok{z.test}\NormalTok{(}\AttributeTok{x=}\NormalTok{mlbsample}\SpecialCharTok{$}\NormalTok{height, }\AttributeTok{mu=}\FloatTok{72.1}\NormalTok{, }\AttributeTok{sigma.x =} \FloatTok{2.31}\NormalTok{, }\AttributeTok{alternative =} \StringTok{\textquotesingle{}less\textquotesingle{}}\NormalTok{ )}
\CommentTok{\# ztest99lefttail  \# p{-}value is large, fail to reject null, which is mu = 73.2 (or greater.)}
\FunctionTok{unloadPkg}\NormalTok{(}\StringTok{"BSDA"}\NormalTok{)}
\end{Highlighting}
\end{Shaded}

\begin{Shaded}
\begin{Highlighting}[]
\CommentTok{\# force this one result to always show, regardless of global settings}
\NormalTok{ztest99lefttail}
\end{Highlighting}
\end{Shaded}

\begin{verbatim}
## 
##  One-sample z-Test
## 
## data:  mlbsample$height
## z = 4, p-value = 1
## alternative hypothesis: true mean is less than 72.1
## 95 percent confidence interval:
##    NA 74.4
## sample estimates:
## mean of x 
##      73.7
\end{verbatim}

In these cases, the Z-intervals produced are one-sided intervals, which
we did not used before. But it is possible to create estimates of
one-sided intervals.

\hypertarget{t-test}{%
\subsection{T-Test}\label{t-test}}

Similar for t-test, except it is \textbf{easier}, since we do not need
to know the standard deviation (sigma) of the population.

\begin{Shaded}
\begin{Highlighting}[]
\CommentTok{\# t.test is included in the basic R package \textquotesingle{}stats\textquotesingle{}}
\NormalTok{ttest95 }\OtherTok{=} \FunctionTok{t.test}\NormalTok{(}\AttributeTok{x=}\NormalTok{mlbsample}\SpecialCharTok{$}\NormalTok{height, }\AttributeTok{mu=}\FloatTok{72.1}\NormalTok{) }\CommentTok{\# default conf.level = 0.95}
\CommentTok{\# t.test(x=mlb$height, mu=73.2, sigma.x = 2.18, conf.level=0.99 )}
\NormalTok{ttest95}
\NormalTok{ttest99 }\OtherTok{=} \FunctionTok{t.test}\NormalTok{(}\AttributeTok{x=}\NormalTok{mlbsample}\SpecialCharTok{$}\NormalTok{height, }\AttributeTok{mu=}\FloatTok{72.1}\NormalTok{, }\AttributeTok{conf.level=}\FloatTok{0.99}\NormalTok{ )}
\CommentTok{\# ttest99}
\FunctionTok{names}\NormalTok{(ttest99)}
\NormalTok{ttest99}\SpecialCharTok{$}\NormalTok{conf.int}
\NormalTok{ttest99}\SpecialCharTok{$}\NormalTok{alternative}
\NormalTok{ttest99}\SpecialCharTok{$}\NormalTok{estimate}
\end{Highlighting}
\end{Shaded}

Again, we can check that the optional argument of \texttt{conf.level}
here only affects the confidence interval result, but not the p-value.

\begin{Shaded}
\begin{Highlighting}[]
\CommentTok{\# force this one result to always show, regardless of global settings}
\NormalTok{ttest99}
\end{Highlighting}
\end{Shaded}

\begin{verbatim}
## 
##  One Sample t-test
## 
## data:  mlbsample$height
## t = 5, df = 29, p-value = 6e-05
## alternative hypothesis: true mean is not equal to 72.1
## 99 percent confidence interval:
##  72.8 74.6
## sample estimates:
## mean of x 
##      73.7
\end{verbatim}

And here is the left- and right-tailed cases:

\begin{Shaded}
\begin{Highlighting}[]
\NormalTok{ttestrighttail }\OtherTok{=} \FunctionTok{t.test}\NormalTok{(}\AttributeTok{x=}\NormalTok{mlbsample}\SpecialCharTok{$}\NormalTok{height, }\AttributeTok{mu=}\FloatTok{72.1}\NormalTok{, }\AttributeTok{alternative =} \StringTok{\textquotesingle{}greater\textquotesingle{}}\NormalTok{ ) }\CommentTok{\# conf.level=0.99, }
\NormalTok{ttestrighttail  }\CommentTok{\# p{-}value is small, reject null, adopt alternative mu is greater than 73.2}
\NormalTok{ttestlefttail }\OtherTok{=} \FunctionTok{t.test}\NormalTok{(}\AttributeTok{x=}\NormalTok{mlbsample}\SpecialCharTok{$}\NormalTok{height, }\AttributeTok{mu=}\FloatTok{72.1}\NormalTok{, }\AttributeTok{alternative =} \StringTok{\textquotesingle{}less\textquotesingle{}}\NormalTok{ ) }\CommentTok{\# conf.level=0.99, }
\CommentTok{\# ttestlefttail  \# p{-}value is large, fail to reject null, which is mu = 73.2 (or greater.)}
\end{Highlighting}
\end{Shaded}

\begin{Shaded}
\begin{Highlighting}[]
\CommentTok{\# force this one result to always show, regardless of global settings}
\NormalTok{ttestlefttail}
\end{Highlighting}
\end{Shaded}

\begin{verbatim}
## 
##  One Sample t-test
## 
## data:  mlbsample$height
## t = 5, df = 29, p-value = 1
## alternative hypothesis: true mean is less than 72.1
## 95 percent confidence interval:
##  -Inf 74.3
## sample estimates:
## mean of x 
##      73.7
\end{verbatim}

I have not found a good way to display these results in better format.
For now, we'll just screen dump these onto the html.

\hypertarget{two-sample-t-test}{%
\subsection{Two-sample t-test}\label{two-sample-t-test}}

Let us now try to use the two-sample t-test, which is a similar way of
using the t-test, but on two different samples. First, create a
different sample.

\begin{Shaded}
\begin{Highlighting}[]
\FunctionTok{set.seed}\NormalTok{(}\DecValTok{999}\NormalTok{) }\CommentTok{\# just so that everyone have the same sample for comparison}
\NormalTok{mlbsample2 }\OtherTok{=}\NormalTok{ mlb[ }\FunctionTok{sample}\NormalTok{(}\FunctionTok{nrow}\NormalTok{(mlb),}\DecValTok{20}\NormalTok{), ]}
\CommentTok{\# compare the two samples}
\FunctionTok{format}\NormalTok{(}\FunctionTok{mean}\NormalTok{(mlbsample2}\SpecialCharTok{$}\NormalTok{height), }\AttributeTok{digits=}\DecValTok{4}\NormalTok{)}
\FunctionTok{format}\NormalTok{(}\FunctionTok{mean}\NormalTok{(mlbsample2}\SpecialCharTok{$}\NormalTok{weight, }\AttributeTok{na.rm =} \ConstantTok{TRUE}\NormalTok{), }\AttributeTok{digits=}\DecValTok{5}\NormalTok{)}
\FunctionTok{format}\NormalTok{(}\FunctionTok{mean}\NormalTok{(mlbsample}\SpecialCharTok{$}\NormalTok{height), }\AttributeTok{digits=}\DecValTok{4}\NormalTok{)}
\FunctionTok{format}\NormalTok{(}\FunctionTok{mean}\NormalTok{(mlbsample}\SpecialCharTok{$}\NormalTok{weight, }\AttributeTok{na.rm =} \ConstantTok{TRUE}\NormalTok{), }\AttributeTok{digits=}\DecValTok{5}\NormalTok{)}
\end{Highlighting}
\end{Shaded}

Next, perform the 2-sample t-test on height and weight

\begin{Shaded}
\begin{Highlighting}[]
\NormalTok{ttest2sample\_height }\OtherTok{=} \FunctionTok{t.test}\NormalTok{(mlbsample}\SpecialCharTok{$}\NormalTok{height,mlbsample2}\SpecialCharTok{$}\NormalTok{height)}
\NormalTok{ttest2sample\_height}
\NormalTok{ttest2sample\_weight }\OtherTok{=} \FunctionTok{t.test}\NormalTok{(mlbsample}\SpecialCharTok{$}\NormalTok{weight,mlbsample2}\SpecialCharTok{$}\NormalTok{weight)}
\NormalTok{ttest2sample\_weight}
\end{Highlighting}
\end{Shaded}

Notice that the two samples do not need to have the same sample size.
Even at level of significance alpha set at = 0.50, we still fail to
reject the null. The null hypothesis is that the two samples carry the
same average.

Imagine if we pull the two samples one from NBA players and one from
WNBA players, do you expect a high or low p-value for the two-sample
t-test?

Now, you can try all the above for the weight data for practice.

\end{document}
